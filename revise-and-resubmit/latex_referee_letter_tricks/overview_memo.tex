
\documentclass[12pt,pdftex, notitlepage]{article}
%%%%%%%%%%%%%%%%%%%%%%%%%%%%%%%%%%%%%%%%%%%%%%%%%%%%%%%%%%%%%%%%%%%%%%%%%%%%%%%%%%%%%%%%%%%%%%%%%%%%%%%%%%%%%%%%%%%%%%%%%%%%
\usepackage[pdftex]{graphicx}
\usepackage[nomessages]{fp}
\usepackage[usenames,dvipsnames]{color}
%\usepackage[notablist,nofiglist]{endfloat}
\usepackage{amsmath, amsthm, amssymb}
%\usepackage{cje}
%\usepackage[colon]{natbib}
\usepackage[hyphens]{url}
\usepackage{fullpage}
\usepackage{bbm}
%\usepackage{type1cm}
\usepackage{setspace}
\usepackage{float}
\usepackage{verbatim}
\usepackage{pdfpages}
\usepackage{lscape}
\usepackage{booktabs}
\usepackage{multirow}
\usepackage[english]{babel}
\usepackage{tabulary}
\usepackage{tabularx}
\usepackage{array}
\usepackage{sectsty}
\usepackage{pdflscape}
\sectionfont{\large}
\usepackage{xfrac}
%\usepackage[small,compact]{titlesec}
 \usepackage[normalem]{ulem}
\newcommand{\sups}[1]{\ensuremath{^{\textrm{#1}}}}
\newcommand{\subs}[1]{\ensuremath{_{\textrm{#1}}}}
\usepackage{pgfplots}
\usepackage{soul}
%\usepackage{hyperref}
\usepackage{caption}
\usepackage{subcaption}
%\captionsetup[subfigure]{position=top}
\usepackage{tabularx}

\usepackage{clipboard}
\newclipboard{unions_stuff}

\usepackage[backend=biber, natbib=true, style=authoryear, firstinits=true, doi=false,isbn=false,url=false]{biblatex}
\DefineBibliographyStrings{english}{%
  references = {\large References},
}

\bibliography{unionbib} 
\renewbibmacro{in:}{}

\usepackage{fancyhdr}

\usepackage{tikz}
\usepackage{graphicx}
\usetikzlibrary{arrows}
%\usepackage[dvips]{color}

\floatstyle{plain}

\let\footnotesize=\small
\let\titlesize=\small

%\linespread{1.3}
\usepackage{geometry}
\geometry{verbose,letterpaper,tmargin=1.0in,bmargin=1in,lmargin=1in,rmargin=1in}

\renewcommand*{\thetable}{\arabic{table}}

\singlespacing

\begin{document}

\author{Ilyana}
\title{\vspace{-2.5cm}\Large{Latex tricks for revisions and response letters}\thanks{Just my two cents.  Feel free to share if you like.}
}
\date{\today}

\maketitle
\begin{abstract}
\noindent
With a few packages, you can quote passages of text and table and figures numbers from one document to another.  These are very useful in revising a paper in response to referee and editor letters, as you can change something in the paper and then just ``call'' that new paragraph or table or figure into the referee letter.  No need to cut or paste from the paper to the response letter and when you revise something in the paper, it will automatically update in the letter.
%Shorter
\end{abstract}

%\vspace{.5cm}
\medskip
\noindent
%\textbf{JEL Classification Numbers}:

\noindent
%\textbf{Key words}: Unions, Inequality, Economic History.

\thispagestyle{empty}

\newpage

\onehalfspacing
\setcounter{page}{1}


%\maketitle
\section{Goal}

You want to make the experience of reading your revision as easy for editors and referees, who probably have not thought of your paper in over a year.  The goal (IMO) is to provide a response memo that is ``self-contained.''  With rare exception, the referee should be able to read the response letter without ever having to flip back to the revised paper.  This strategy makes the response letter quite long, but it creates less work and annoyance for the referee/editor.  

\section{Examples of self-contained content}

\begin{itemize}

\item Suppose the referee asked you to add additional discussion of robustness checks.  Instead of merely writing in the response letter something like: `this discussion can be found in Section 4.2' and expecting him/her to find the passage, I prefer to simply \textit{quote} the material directly into the response letter.  Below I show how you can automate that quotation (i.e., the quoted passage in the response letter will automatically change if you change the passage in the paper).

\item Suppose you want to reference a table or figure number in the paper in the response letter.  Again, you can automate this process so that if the order of figures or tables changes in the paper, the numbering will automatically change in the response letter.

\item Before delving point-by-point into the referee reports, I think it is useful to 


\section{Packages you will need}

You will need a few packages that allow separate .tex files to ``speak'' to each other:

\begin{itemize}
\item The \texttt{clipboard} package allows you to copy and paste from one document to another or within the same document.

\item The \texttt{xr} package allows you to cross reference the figure and table numbers from one document (typically the revised paper) to another (typically the response letter).

\end{itemize}

To make life easier for the editors and referees, I also have separate headings in the response memo so that they can just flip to the part of the document that is address to them. This functionality requires the \texttt{fancyhdr} package.

Finally, and often the most painful, I create separate bibliographies for each part of the response memo.  I often find that a majority of the latex bugs arise in this part of the process, so it's the part you might want to skip.  

To get the references in the letters to work, you will need to choose biber as your bibtex engine (not bibtex, which I think is the default).  I then separate each part (response to the editor, response to each referee) in the response memo into a ``refsection.''  Then you get separate bibliographies for each part of the document.

This feature is also very useful in the main paper if you want separate bibliographies for the main paper and for appendix sections (if you only cite a source in the appendix, then the citation should not appear in the bibliography for the main paper).

\section{Working ``fake'' example}

The first thing to do is to copy-and-paste the text from the editor letter and referee reports into a \LaTeX\ file.  \textit{Watch out for weird copy-paste errors, as copying from PDF and even Word often messes up punctuation in the .tex document.}

In this folder you will see a fake paper (\texttt{fake\_paper.tex}) and fake response document (\texttt{fake\_response\_memo.tex}).  I annotate these files with comments, highlighting where I use the various features.  There is also a .bib file (\texttt{fakebib2.bib}).

Note that you have to compile the documents in a certain order.  You must compile the paper first (Latex, Biber, Latex, Latex).  This step creates all the various auxilary files that the response-memo document needs.  Then you compile the response memo (Latex, Biber, Latex, Latex).



\section{Some little hacks}

One thing I noticed is that sometimes the  \texttt{xr} package clashes with the \texttt{hyperlinks} package.  So I just shut off the \texttt{hyperlinks} package in the R\&R stage (you can always bring it back for the final paper when you are no longer linking the paper to a response letter).

Sometimes a journal wants \textit{separate} letters for each referee and editor.  I think the easiest way to handle this request is just to split the long .pdf document created by \texttt{fake\_response\_memo.tex} into separate documents at the very end with Acrobat or whatever .pdf reader you use.  I think it's annoying to have in some cases up for four or five separate response .tex files running around.

As I said, the fiddliest part of the process is the bibliographies for each response letter.  You could just not include a bibliography at all for your response letters.  

\section{In conclusion....}

This process works for me but surely there are more elegant solutions (and feel free to share any that you come up with).  Congratulations on your R\&R!

\end{document}
